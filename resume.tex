\documentclass[a4paper]{article}
\usepackage{fullpage}
\usepackage{amsmath}
\usepackage{amssymb}
\usepackage{textcomp}
\usepackage[utf8]{inputenc}
\usepackage[T1]{fontenc}
\usepackage{scrextend}
\usepackage{graphicx,calc}
\usepackage{fontawesome}
\usepackage[usenames,dvipsnames]{xcolor}
\usepackage[
    bookmarks, colorlinks, breaklinks, unicode,
    pdftitle={Jonathan Ross - Resume},
    pdfauthor={Jonathan Ross},
    pdfproducer={http://donahut.github.io}
]{hyperref}

\textheight=10in
\pagestyle{empty}
\raggedright
\hypersetup{linkcolor=blue,citecolor=blue,filecolor=black,urlcolor=MidnightBlue}
\def\bull{\vrule height 0.8ex width .7ex depth -.1ex }
\pagestyle{plain}

% DEFINITIONS FOR RESUME %%%%%%%%%%%%%%%%%%%%%%%

\newcommand{\area} [2] {
    \vspace*{-9pt}
    \begin{verse}
        \textbf{#1}   #2
    \end{verse}
}

\newcommand{\lineunder} {
    \vspace*{-8pt} \\
    \hspace*{-18pt} \hrulefill \\
}

\newcommand{\header} [1] {
    {\hspace*{-18pt}\vspace*{6pt} \textsc{#1}}
    \vspace*{-6pt} \lineunder
    \vspace{2mm}
}

\newcommand{\employer} [3] {
    { \textbf{#1} (#2)\\ \underline{\textbf{\emph{#3}}}\\  }
}

\newcommand{\contact} [3] {
    \vspace*{-10pt}
    \begin{center}
        {\Huge \scshape {#1}}\\
        #2 \\ #3
    \end{center}
    \vspace*{-8pt}
}

\newenvironment{achievements}{
    \begin{list}
        {$\bullet$}{\topsep 0pt \itemsep -2pt}}{\vspace*{4pt}
    \end{list}
}

\newcommand{\schoolwithcourses} [4] {
    \textbf{#1} #2 $\bullet$ #3\\
    #4 \\
    \vspace*{5pt}
}

\newcommand{\school} [4] {
    \textbf{#1} #2 $\bullet$ #3\\
    #4 \\
}
% END RESUME DEFINITIONS %%%%%%%%%%%%%%%%%%%%%%%



\begin{document}
\vspace*{-50pt}

    

%==== Profile ====%

\begin{minipage}[t]{0.5\textwidth}
    \vspace{-24pt}
    {\Huge\scshape{Jonathan Ross}}\\
    Boston, MA
\end{minipage}
\begin{minipage}[t]{0.48\textwidth}
    \flushright 
    \begin{tabular}[h]{ll@{}}
        \faEnvelope & \href{mailto:golgobot@gmail.com}{golgobot@gmail.com}\\
        \faPhone    & 857-383-9355\\
        \faGithub   & \href{http://www.github.com/golgobot}{github.com/golgobot}\\
        \faLinkedin & \href{http://www.linkedin.com/in/jonathan-louis-ross/}{linkedin.com/in/jonathan-louis-ross}
    \end{tabular}
\end{minipage}


\vspace{8mm}

\setlength{\parindent}{4ex}
Proven leader and software architect with a history of delivering large, complex, interdisciplinary
products. Effective at leading and mentoring large teams and driving consensus and collaboration
across vastly different technology stacks. Takes a business-minded, solutions-driven approach to
technical design.

\setlength{\parindent}{0ex}
\vspace{4mm}

%==== Technical Skills ====%
\header{\faGear \hspace{1pt} Technical Skills}
\begin{tabular}[h]{@{}ll}
Languages: &            C++, TypeScript, JavaScript, Java, Python\\[4pt]
Graphics: &             Qt/QML/WebEngine, GStreamer, Ogre3D, Pixi.js, Three.js\\[4pt]
Vision: &               OpenCV\\[4pt]
Frameworks/Libs: &      STL, Boost (msm, mpl), PEGTL, Eigen, Poco, ALSA, v8\\[4pt]
Embedded: &             Android OS, Buildroot, Arduino, Raspberry Pi\\[4pt]
Web: &                  Emscripten, HTML5/CSS/JS, React/Redux, Webpack\\[4pt]
Cloud: &                Node.js, Tomcat, AWS, Docker, Microservices Architecture\\[4pt]
Tools: &                Git, GitHub, Jira, Agile (Scrum)\\[4pt]
Build: &                CMake, Make, QMake, Gulp
\end{tabular}

\vspace{4mm}

%==== Experience ====%
\header{\faBriefcase \hspace{1pt} Experience}
\vspace{1mm}

\textbf{\textsc{Amazon Lab126}} \hfill Remote\\
\vspace{2mm}

\begin{addmargin}[1em]{0em}
    \textit{Senior Software Engineer} \hfill September 2022 - Present\\
    \vspace{-1mm}
    \begin{itemize} \itemsep 1pt
        \item Engineering lead for the \href{https://www.amazon.com/dp/B078NSDFSB}{Amazon Astro}, an Alexa enabled mobile robot.
        \item Currently leading a cross org effort to create a consistent and more dynamic character experience
            across all features through an architectural redesign.
        \item Developing and designing a robot centric programming language for expressing character.  
    \end{itemize}

\end{addmargin}


\textbf{\textsc{Amazon Alexa Devices}} \hfill Cambridge, MA\\
\vspace{2mm}

\begin{addmargin}[1em]{0em}
    \textit{Senior Software Engineer} \hfill September 2018 - September 2022\\
    \vspace{-1mm}
    \begin{itemize} \itemsep 1pt
        \item Lead architect of the \href{https://www.amazon.com/dp/B07VHZ41L8/}{Echo Show 10's} motion capabilities.
        \begin{itemize}
            \item Built out all motor/motion features, safety features, motion policies, vision and audio integrations.
            \item Designed and built automatic motion CX for all Alexa experiences.
            \item Worked with multiple internal teams to integrate 1p skills: Alexa Guard, Drop in, Alexa Video Calls, Photos, and Alexa's demo team.
        \end{itemize}
        \item Designed the specification for and built out 3p APIs for Echo Show 10s new capabilities, which allows
        3p developers to sense and react to the environment in their skills. This includes
        \begin{itemize}
            \item APL extensions for \href{https://developer.amazon.com/de-DE/docs/alexa/alexa-presentation-language/apl-ext-smart-motion.html}{Smart Motion} 
            and \href{https://developer.amazon.com/de-DE/docs/alexa/alexa-presentation-language/apl-ext-entity-sensing.html}{Entity Sensing} APL extensions.
            \item Web API for Games \href{https://developer.amazon.com/en-US/docs/alexa/web-api-for-games/alexa-games-extensions-smartmotion.html}{Smart Motion}
            and \href{https://developer.amazon.com/en-US/docs/alexa/web-api-for-games/alexa-games-extensions-entitysensing.html}{Entity Sensing} Extensions.
        \end{itemize}
        \item Architected the \href{https://developer.amazon.com/en-US/docs/alexa/alexa-presentation-language/test-apl-skills-dev-console.html#use-the-smart-motion-simulator}{Echo Show 10's} 
        web simulator, built with Emscripten.
        \item Main developer and architect of the web \href{https://developer.amazon.com/en-US/docs/alexa/alexa-presentation-language/understand-apl.html}{APL} renderer, 
            used to display APL on TVs, XBox, and in the \href{https://developer.amazon.com/alexa/console/ask/displays}{APL authoring tool} and Alexa web simulator.
    \end{itemize}

\end{addmargin}

\newpage

\textbf{\textsc{Jibo, Inc}} \hfill Boston, MA\\
\vspace{2mm}

\begin{addmargin}[1em]{0em}
    \textit{Chief Architect} \hfill April 2018 - June 2018\\
    \vspace{-1mm}
    \begin{itemize} \itemsep 1pt
        \item Technical lead for a team of 50 engineers.
        \item Set technology vision and direction for the company.
        \item Converted business objectives into a unified technology plan.
        \item Sole technical stakeholder for all product decisions.
        \item Main point of contact for all technology integrations with external
            business partners.
        \item Restructured team and appointed architectural leads to most effectively
            cover all major components of the tech stack.
        \item Designed large key components including dynamic proactivity, reminders, 3rd party
            cloud push, 3rd party cloud integration and authentication models, 
            sandboxed cloud-hosted on-robot skills, and hybrid cloud/on-robot skills.
        \item Spearheaded unification of cloud infrastructure to enable faster cross stack
            development of new features.
        
    \end{itemize}
    \textit{Chief Robot Architect} \hfill November 2017 - April 2018\\
    \vspace{-1mm}
    \begin{itemize} \itemsep 1pt
        \item Lead a team of 12 robotics and vision engineers in a total rewrite of 
            Jibo's embedded software stack, called ``Project Phoenix''.
        \item Set overall architectural direction of Phoenix and developed
            a transition plan to fast track it to production.
        \item Lead developer for Jibo's new graphics system built using Qt.
        \item Technical lead for all embedded development including v1 systems.
    \end{itemize}
    \textit{Head of SDK} \hfill Jan 2015 - November 2017\\
    \vspace{-1mm}
    \begin{itemize} \itemsep 1pt
        \item Overall architect and SDK team lead.
        \item Hired, built, and led a talented team of 5 engineers and 2 QA testers 
            responsible for the development of Jibo’s SDK, which include visual behavior 
            editors, NLU and dialog tools, animation tools, and a robot simulator.
    \end{itemize}

    \textit{Software Architect} \hfill November 2012 - Jan 2015\\
    \vspace{-1mm}
    \begin{itemize} \itemsep 1pt
        \item Second employee and first engineer at Jibo.
        \item Ideated and pitched concepts/demos to investors.
        \item Responsible for building 2 prototype robots, one of which starred in Jibo's
            Indiegogo campaign.
        \item Worked full stack, writing everything from microprocessor firmware to 
            high level behavioral engines.
    \end{itemize}
\end{addmargin}

%==== Zynga =====%
\textbf{\textsc{Zynga}} \hfill San Francisco, CA\\
\vspace{2mm}

\begin{addmargin}[1em]{0em}

\textit{Principal Software Engineer} \hfill May 2011 - October 2012\\
\vspace{-1mm}
\begin{itemize} \itemsep 1pt
    \item Tech lead and server and client side engineer for ChefVille.
    \item Developed RAD, a UI framework that became the standard at Zynga and localized
    into 18 different languages, including languages read right to left.
\end{itemize}
\textit{Senior Software Engineer} \hfill March 2011 - May 2011\\
\vspace{-1mm}
\begin{itemize} \itemsep 1pt
    \item Client and server side engineer for CafeWorld and CityVille.
\end{itemize}
\end{addmargin}

\newpage

%==== Disney =====%
\textbf{\textsc{Disney}} \hfill Los Angeles, CA\\
\vspace{2mm}

\begin{addmargin}[1em]{0em}

\textit{Senior Software Engineer} \hfill November 2007 - March 2010\\
\vspace{-1mm}
\begin{itemize} \itemsep 1pt
    \item Developed and maintained high performance real-time server side
        technologies for current and unannounced virtual worlds.
    \item A lead developer on \textit{World of Car Online}, a Flash based 3D MMO for kids.
        Developed custom server and client rigid body physics engine, hand-tuned for low 
        end machines; AI and an AI scripting system; single and multi-player Circuit Racing;
        and the race career and treadmill system. Also co-wrote the game's questing and 
        questing scripting system.
    \item Inventor of ToyBridge, a framework for communication between a web 
        deployed Flash application and hardware devices.
    \item Member of ToyMorrow, an interdivisional high-tech toys of the future think 
        tank.
\end{itemize}

\end{addmargin}

%==== Xplana =====%
\textbf{\textsc{Xplana Learning}} \hfill Boston, MA\\
\vspace{2mm}

\begin{addmargin}[1em]{0em}

\textit{Senior Software Engineer} \hfill June 2006 - November 2007\\
\vspace{-1mm}
\begin{itemize} \itemsep 1pt
    \item Led a team of client side engineers in developing online learning software.
    \item Developed an online customizable ebook, and math learning courses, which 
        won a Codie award for ``Best Mathematics Instructional Solution.''
    \item Participated in overall company strategy, and pushed new ideas and technology 
        to keep products on the cutting edge.
\end{itemize}

\end{addmargin}

%==== Lab =====%
\textbf{\textsc{Nicolelis Neuroscience Lab at Duke University}} \hfill Durham, NC\\
\vspace{2mm}

\begin{addmargin}[1em]{0em}

\textit{Research Engineer} \hfill May 2004 - May 2006\\
\vspace{-1mm}
\begin{itemize} \itemsep 1pt
    \item Provided mathematical analysis of brain waves.
    \item Wrote programs to aid in the visualization of these data.
    \item Developed an algorithm to automatically detect brain states 
        in rats (REM, Slow Wave, Awake) base on brain signals.
    \item Implemented artificial neural nets to show and imitate the 
        mathematical properties of dreams.

\end{itemize}

\end{addmargin}

\header{\faLightbulbO \hspace{1pt} Patents}

\textbf{System to mitigate image jitter by an actuator driven camera}\\
United States Patent No. 11,265,469\\
Granted March 1st, 2022
\vspace*{2mm}

\textbf{Persistent companion device configuration and deployment platform}\\
United States Patent Application 20,170,206,064\\
Filed March 30th, 2017
\vspace*{2mm}

\textbf{Apparatus and methods for providing a persistent companion device}\\
United States Patent Application 20,150,314,454\\
Filed July 15th, 2015\\

\vspace*{2mm}
\textbf{System and Method for Integrated Hardware Platform for Flash Applications
with Distributed Objects}\\
Patent No.: US 8,924,989 B2\\
Granted December 30th, 2014 \\
\vspace*{4mm}

\newpage

\pagebreak[3]

\header{\faCommentingO \hspace{1pt} Speaking Events}
\textbf{Babson} \hfill November 2017 - Present\\
Recurring guest lecturer for entrepreneurship classes 
\vspace*{2mm}

\textbf{Github Universe} \hfill February 2016\\
Building a Social Robot with Atom and Electron 
\vspace*{2mm}

\textbf{SpeechTek} \hfill August 2015\\
Building Skills for a Conversational Robot
\vspace*{2mm}

\pagebreak[3]

\textbf{Exploring Computer Science} \hfill April 2010\\
\begin{addmargin}[0em]{4em}
Program targeting high school students meant to increase the number of
minorities and women who gain exposure to engineering
\end{addmargin}
\vspace*{2mm}

\textbf{USC GamePipe Laboratory} \hfill November 2009\\
Speaker for seminar series on game AI and physics 
\vspace*{4mm}

\header{\faTrophy \hspace{1pt} Awards}

\textbf{Best Invention of 2017} \hfill Time Magazine\\
Time's annual best inventions issue awarded Jibo best invention of 2017
\vspace*{2mm}

\textbf{CTO Award} \hfill Zynga\\
For the development of the RAD UI framework 
\vspace*{2mm}

\textbf{Disney Inventor Award} \hfill Disney\\
Walt Disney Inventor Award for ToyBridge
\vspace*{4mm}

%==== Education ====%
\header{\faGraduationCap \hspace{1pt} Education}
\textbf{Duke University}\hfill Durham, NC\\
BS Electrical Engineering\\
\vspace{4mm}

\end{document}